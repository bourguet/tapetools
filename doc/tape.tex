% tape.tex
% Copyright (c) 2004, 2009  Jean-Marc Bourguet
% 
% All rights reserved.
% 
% Redistribution and use in source and binary forms, with or without
% modification, are permitted provided that the following conditions are
% met:
% 
% * Redistributions of source code must retain the above copyright notice,
% this list of conditions and the following disclaimer.
% 
% * Redistributions in binary form must reproduce the above copyright
% notice, this list of conditions and the following disclaimer in the
% documentation and/or other materials provided with the distribution.
% 
% * Neither the name of the <ORGANIZATION> nor the names of its
% contributors may be used to endorse or promote products derived from this
% software without specific prior written permission.
% 
% THIS SOFTWARE IS PROVIDED BY THE COPYRIGHT HOLDERS AND CONTRIBUTORS "AS
% IS" AND ANY EXPRESS OR IMPLIED WARRANTIES, INCLUDING, BUT NOT LIMITED TO,
% THE IMPLIED WARRANTIES OF MERCHANTABILITY AND FITNESS FOR A PARTICULAR
% PURPOSE ARE DISCLAIMED. IN NO EVENT SHALL THE COPYRIGHT HOLDER OR
% CONTRIBUTORS BE LIABLE FOR ANY DIRECT, INDIRECT, INCIDENTAL, SPECIAL,
% EXEMPLARY, OR CONSEQUENTIAL DAMAGES (INCLUDING, BUT NOT LIMITED TO,
% PROCUREMENT OF SUBSTITUTE GOODS OR SERVICES; LOSS OF USE, DATA, OR
% PROFITS; OR BUSINESS INTERRUPTION) HOWEVER CAUSED AND ON ANY THEORY OF
% LIABILITY, WHETHER IN CONTRACT, STRICT LIABILITY, OR TORT (INCLUDING
% NEGLIGENCE OR OTHERWISE) ARISING IN ANY WAY OUT OF THE USE OF THIS
% SOFTWARE, EVEN IF ADVISED OF THE POSSIBILITY OF SUCH DAMAGE.

\documentclass[a4paper,12pt]{article}

\usepackage[latin9]{inputenc}
\usepackage[T1]{fontenc}
\usepackage{lmodern}

\title{Tape formats for PDP-11 simulators}
\author{Jean-Marc Bourguet}
\date{17 october 2004}

\newcommand{\file}[1]{\texttt{#1}}
\newcommand{\prog}[1]{\texttt{#1}}

\begin{document}

\maketitle

\section{Introduction}

When wanting to manipulate tapes for the PDP-10 on disk and on other
systems, there are two problems to solve:
\begin{enumerate}
\item representing the tapes as file on disk.  Tapes are record oriented
  devices (the transfer unit is a record of varying length) while files on
  current systems are not structured in record.
\item representing 36 bits data on 9 tracks tapes.  The PDP-10 is a 36 bits
  computer and the tapes have 9 tracks used to store 8 bits bytes.
  Software on the PDP-10 used several methods to do the packing.
\end{enumerate}

\section{Representing tapes as file on disk}

\subsection{TPC}

This is the simplest representation.  The data for each record is preceeded
by two bytes in the less significant byte first order representing the
length of the record.  If this length is odd, a padding byte (of undefined
value) is added before the next record.

A tape mark (indication of end of file, two tape marks are used to mark the
end of the recorded part of the tape) is an empty record.

\subsection{TPE}

In this representation, the record is preceeded by four bytes representing
the length of the record (anew in little endian way).  If the most
significant bit of the header is 1, then there was an error in the record.
The record is not padded but it is followed by a copy of the header so that
simulating backward tape motion is easy.

\subsection{TPS}

This is nearly the same format as TPE, but there is a padding byte for odd
length record.

\section{Representing 36 bits data on 9 tracks tapes}

\subsection{Core-Dump}

36 bits data is represented as 5 bytes.  The first four contains the 32
most significant bits in most significant byte first order.  The last byte
contains the 4 less significant bits padded with 0 in the 4 most
significant bits of the result.

\begin{verbatim}
                B0  1  2  3  4  5  6  7
                 8  9 10 11 12 13 14 15
                16 17 18 19 20 21 22 23
                24 25 26 27 28 29 30 31
                 0  0  0  0 32 33 34 35
\end{verbatim}

\subsection{Industry-Compatible}

The 32 most significant bits are represented as 4 bytes.  The four
remaining bits are ignored.

\begin{verbatim}
                B0  1  2  3  4  5  6  7
                 8  9 10 11 12 13 14 15
                16 17 18 19 20 21 22 23
                24 25 26 27 28 29 30 31
\end{verbatim}

\subsection{Sixbit}

Groups of 6 bits are output.

\begin{verbatim}
                 0  0 B0  1  2  3  4  5
                 0  0  6  7  8  9 10 11
                 0  0 12 13 14 15 16 17
                 0  0 18 19 20 21 22 23
                 0  0 24 25 26 27 28 29
                 0  0 30 31 32 33 34 35
\end{verbatim}

\subsection{ANSI-ASCII}

\begin{verbatim}
                 0 B0  1  2  3  4  5  6
                 0  7  8  9 10 11 12 13
                 0 14 15 16 17 18 19 20
                 0 21 22 23 24 25 26 27
                35 28 29 30 31 32 33 34
\end{verbatim}

With the less significant bit beeing 0 in ASCII files on the PDP-10, this
allow easy transfer of text data.

\subsection{High-density}

\begin{verbatim}
                B0  1  2  3  4  5  6  7
                 8  9 10 11 12 13 14 15
                16 17 18 19 20 21 22 23
                24 25 26 27 28 29 30 31
                32 33 34 35 B0  1  2  3
                 4  5  6  7  8  9 10 11
                12 13 14 15 16 17 18 19
                20 21 22 23 24 25 26 27
                28 29 30 31 32 33 34 35
\end{verbatim}

This has an higher density and is also used by the FTP protocol (RFCxxx).

\section{References}

All the information in this document, and then some more is contained in
the \file{vtape.txt} file in the \prog{klh} distribution.  I sadly found
out this only after having examined the \prog{simh} source and having
experimented with setting SET TAPE FORMAT and examaning the result.

\end{document}
